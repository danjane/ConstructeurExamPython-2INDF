\documentclass[12pt]{exam}
\pointsinmargin

\usepackage[utf8]{inputenc}
\usepackage{listings}
\lstset{language=Python, basicstyle=\ttfamily, numbers=left, numberstyle=\tiny}
\usepackage[margin=2cm]{geometry}

\begin{document}

\begin{center}
\textbf{2INDF Examen de Programmation Python} \hfill Nom : \rule{6cm}{0.4pt}
\end{center}

\begin{questions}

\question[4] Complétez la table de traçage suivante pour le code donné :

\hfill
\begin{minipage}[t]{0.2\textwidth}
\vspace{0mm}
\begin{lstlisting}
VARIABLE0 = BIGINT
VARIABLE1 = SMALLINT
VARIABLE2 = VARIABLE0 PM VARIABLE1
VARIABLE0 = VARIABLE0 * 2
VARIABLE1 = VARIABLE2 PM 1
\end{lstlisting}
\end{minipage}
\hfill
\begin{minipage}[t]{0.4\textwidth}
\vspace{-4mm}
    \begin{tabular}{|c|p{5mm}|p{5mm}|p{5mm}|}
        \hline
        {\footnotesize Ligne} & \texttt{VARIABLE0} & \texttt{VARIABLE1} & \texttt{VARIABLE2} \\
        \hline
        1 &  &  &  \\
        \hline
        2 &  &  &  \\
        \hline
        3 &  &  &  \\
        \hline
        4 &  &  &  \\
        \hline
        5 &  &  &  \\
        \hline
    \end{tabular}
\end{minipage}

AFFECTATIONQUESTIONS

\question[4] Écrivez un programme utilisant une boucle \texttt{for} qui affiche TASKLOOP1.
\fillwithgrid{24mm}

\question[4] Écrivez un programme utilisant une boucle \texttt{while} qui affiche TASKLOOP2.
\fillwithgrid{32mm}

\newpage
\question[5] Écrivez un programme qui affiche TASKCONDITIONNELLE :
\begin{itemize}
    \item TASKC1.
    \item TASKC2.
    \item TASKC3.
\end{itemize}
Vous devez utiliser les instructions conditionnelles et une variable nommée \texttt{TASKVAR} (que l'on supposera déjà affectée).
\smallskip\fillwithgrid{45mm}

\question[11] Voici une liste :
\begin{lstlisting}[language=Python]
prenoms = [`NAME0',`NAME1',`NAME2',`NAME3']
\end{lstlisting}
Écrivez un programme pour effectuer chacune des tâches suivantes. Vous devez utiliser la liste pour réaliser les tâches !

a. Affichez le NIEME prénom :
\fillwithgrid{12mm}

b. Testez à l’aide de l’instruction \texttt{if} et \texttt{in} s’il existe un prénom \texttt{`TEST0'} ou \texttt{`TEST1'}. Si oui, il faut afficher \texttt{`Oui'}, sinon on ne fait rien.
\fillwithgrid{12mm}

c. Afficher tous les prénoms de la liste à l’aide d’une boucle.
\fillwithgrid{12mm}

d. Remplacez \texttt{`NAMEREPLACED'} dans la liste par le prénom \texttt{`Isabelle'}.
\fillwithgrid{12mm}

f. Ajoutez \texttt{`Roxane'} à la fin de la liste.
\fillwithgrid{12mm}

\end{questions}

\end{document}
